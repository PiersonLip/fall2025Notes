\documentclass{article}
\usepackage{darkmode}
\usepackage{amsmath}
\usepackage{amssymb}

\enabledarkmode
\title{Astro Notes}
\author{Pierson Lipschultz}

\begin{document}
\maketitle

\section{Class overview}
Office hours tues and thurs 11-12 astro 237

ta office hours

wed 3:30:530 astro 2367

hw due wed  

hw posted week before

\begin{list}{-}{}
\item solar system 
\item steller evo
\item compact objects
\item galaxy quasi darkmatter
\item cosmic web
\item big bang
\end{list}

course goals
\begin{list}{-}{}
\item apply pys to universe
\item understand foundations of modern astro, astrophys, and cosmology
\item conceptual understanding of the uni based on physical principles
\end{list}

\section{Early Astronomy}
\subsubsection{Greek}
\begin{list}{-}{}
\item Aristotle 
    \begin{list}{-}{}
    \item earth is spherical
    \item partial lunar eclipses
    \item some stars visible from southern locations but not northern and vice versa
    \item had ideas regarding perfect geo influenced by Pythagoras and Plato
    \end{list}
\item Aristarchus (310-230 BC):
\begin{list}{-}{}
\item unpreceded heliocentric framework
\item trig distances earth-moon-sun system
\item angular diameters $\theta_{sun} \approx \theta_{moon} \therefore \frac{A}{C} = \frac{D_{moon}}{D_{moon}}$
\item diameters from lunar eclipses $D_{moon} < D_{earth}$
\end{list} 
\item Eratoshs (176-195 BC):
\begin{list}{-}{}
\item Determined radius of spherical earth $R_{E}$
\item Sun at zenith at noon on summer solstice at Aswan 
\item But further north in Alexandria, Egypt, the sun is south of the zenith by angle $\alpha$
\end{list}
\item Hipparchus (190-120 BC):
\begin{list}{-}{}
\item Discover precession of the equinoxes from examination of star catalogs over centuries
\item established the magnitude system
\end{list}
\item Copernicus (1473-1543):
\begin{list}{-}{}
\item heliocentric
\item earth rotates
\item still assumed uniform circular celestial motion 
\item inferior planets: orbit smaller than earths
\item superior planets: orbits larger than earths
\end{list}
\end{list}

\subsection{\textbf{\large Emergence of modern Astro}}

\textbf{\large Inferior planets}

\begin{list}{-}{}
\item B/C = sin $\theta_{E}$
\item B=C Sin $\theta_{E}$
\item C is AU 
\item Early astronomers didnt know C, so they could only infer rations of B/C. Ie. Orbital radii measured in AU
\end{list}

\noindent
\textbf{\large Superior Planets}

\begin{list}{-}{}
\item Measure time between opposition and eastern quadrature
\item want angle $\theta$ between opp and east quad 
\item $\theta = (\omega_{E} - \omega_{p})$ and $C/B =cos\theta$
\item measure $\tau$ and synodic period, calculate sidereal period and $\omega_{p}$; know $\omega_{E}$ and infer $C/B$
\end{list}

\noindent \textbf{\large Galilean Revolution}
\begin{list}{-}{}
\item Galileo Galilei (1564 -1642)
\item \begin{list}{-}{}
\item improved and used a basic refracting telescoping

\end{list}
\item def publication of early results 1610 \textit{"starry messenger"}
\item \begin{list}{-}{}
\item Moon is cratered; not a perfect Sphere
\item milkyway is made out of stars
\item Jupiter has moons (or as he thought, stars)
\item measured phases of Venus 
\end{list}
\end{list}

\noindent \textbf{\large Phases of Venus}
\begin{list}{-}{}
\item direct confrontation with Ptolemaic geocentric models
\item in Ptolemaic models you only see crescent phases 
\end{list}

\noindent \textbf{\large Tycho Brahe (1546-1601)}
\begin{list}{-}{}
\item Denmark, later Prague
\item Given island by king Fredrick (and staff)
\item made a accurate and vast database of celestial motion
\item had a lead nose?
\item Threw giant ragers
\item supernova named after him 
\end{list}

\noindent \textbf{\large Johannes Kepler (1571–1630, Prague)}
\begin{list}{-}{}
\item 'Inherited' (maybe stole) Brahe's data 
\item also has a SN
\item Kepler fit a new empirical model of heliocentric orbits, abandoning perfect circles 
\begin{list}{-}{}
\item ``\textit{It was as if I awoke from sleep and saw a new light}'' (Kepler, New astronomy)
\end{list}
\end{list}

\noindent \textbf{\large Kepler's Laws}

\noindent \textbf{First law}

\begin{list}{-}{}
\item The planets travel on elliptical orbits with the sun at one focus
\item Semimajor axis, half the major axis 
\item eccentricity: how elliptical (stretched) an orbit is - distance between foci divided by major axis.
\end{list}

\noindent \textbf{second law }

\begin{list}{-}{}
\item A line drawn from the sun to a planet sweeps out equal areas in equal time intervals'
\item perihelion: orbital point closet to the sun
\item aphelion: furthest orbital point from the sun
\end{list}

\noindent \textbf{third law }

Def: \textit{The square of the sidereal orbital periods of the planets are prop to the cubes of the Semimajor axis of their orbits}
\[p^2 = Ka^3\]
\begin{center}
    P = planets sidereal period\\
    a= length of semimajor axis\\
    K = constant
\end{center}

\noindent \textbf{\large Consequences of heliocentric model}
\begin{list}{-}{}
\item retrograde motion of outer planets
\item positions of outer and inner planets wrt sun
\item annual parallax 
\item aberration of starlight
\item Coriolis effect
\end{list}

\noindent \textbf{\large Parallax}
\begin{list}{-}{}
\item annual parallax: change in the apparent position when seen from two diff locations due to earth revolving around the sun. First measured by Bessel in 1838
\end{list}

\noindent \textbf{\large Aberration of starlight}
\begin{list}{-}{}
\item deflection of apparent stellar positions in the direction of the observers motion
\item analog: running throw rain and getting wet in the front and not in the back
\item detected (Picard, 1680); explained (Bradley, 1729)
\item telescope is moving along orbital vector around the sun; translation along orbit cannot exceed transit time of light through telescope
\end{list}

\noindent \textbf{\large Corilois effect: evidence of erath rotation}
\begin{list}{-}{}
\item coriolis acceleration is perp to the direction of motion 
\item \[\vec{a_{cor}}= s\vec{v} \times \vec{\omega}\]
\item can be deduced from a pendulum
\item and in hurricanes!
\end{list}

\pagebreak
\section{Glossary}

\textbf{\large Synodic period }

\begin{list}{-}{}
\item time elapsed between success conjunctions or oppositions
\item this is the period we observe from earth, which is moving
\end{list}
\noindent
\textbf{\large Sidereal Period} 
\begin{list}{-}{}
\item elapsed time of full orbit relative to the fixed stars (inertial ref frame)
\item This is the one we will want to put in keplers laws
\end{list}
\end{document}